
\section{A1}

\subsection{A1.1}

\subsubsection{a}

We need to identify all the sources of non-determinism in the model. 

At the guard create1 in the client module, there's non-determinism. The same applies to the guard create2.
\begin{verbatim}
// Create a new job - length chose non-deterministically
  [create1] state1=0 -> (state1'=1) & (task1'=1);
  [create1] state1=0 -> (state1'=1) & (task1'=2);
  [create1] state1=0 -> (state1'=1) & (task1'=3);
  [create1] state1=0 -> (state1'=1) & (task1'=4);
  [create1] state1=0 -> (state1'=1) & (task1'=5);
\end{verbatim}
This is due to local non-determinism because this doesn't depend on the concurrent execution of two modules.

In the module scheduler at the guard create1 and create2 there's non-determinism.
\begin{verbatim}
// Place a new job at the end of the queue
  [create1] job2=0 -> (job2'=1);
  [create2] job2=0 -> (job2'=2);
\end{verbatim}
This is due to local non-determinism because this doesn't depend on the concurrent execution of two modules.