
\section{C1}

The pairing of task and server is done in a two-step lottery:
one lottery for the first server, and another lottery for
the second server with the remaining tasks, if any.
This will enable both the case where the servers pick
tasks probabilistically, as well as the case where one
server only serves the tasks that has the highest priority
among the tasks at that point in time.

For simplicity's sake, we only have two priority levels, namely 1 and 2.
The priority affect the number of tickets for a task by multiplication
with the priority, such that tasks with priority 2 have double the number
of tickets compared to tasks with priority 1, assuming the tasks are eligible
for scheduling (being in the queue, having non-zero time left, etc.).

If there is one eligible task for scheduling, the first server is always picked.
If there are two tasks, one task is given to each server.
The case with three tasks is the only case where it matters which task gets picked,
because one can be left out.

There are two versions; one for the probabilistic choosing,
and one with a server reserved for high-priority tasks.
See appendix C1 for the code.

We start out with some general properties.
The first properties checks starvation.
The three properties below checks whether it always holds,
that whenever a client has a task,
that task will almost certainly be finished.

\begin{verbatim}
P>=1 [G (state1=1 => !P<=0 [F state1=0]) ]: true.
P>=1 [G (state1=1 => !P<=0 [F state1=0]) ]: true.
P>=1 [G (state1=1 => !P<=0 [F state1=0]) ]: true.
\end{verbatim}

Then we check that it generally holds true that
the probability for eventually starting a task is non-zero.

\begin{verbatim}
P>=1 [G (P>0 [F state1=1]) ]: true.
P>=1 [G (P>0 [F state2=1]) ]: true.
P>=1 [G (P>0 [F state3=1]) ]: true.
\end{verbatim}

All the properties hold for both versions of the model.

Some other properties is safety properties: An example is that
it should always hold that the servers are never serving the same task:

\begin{verbatim}
P>=1 [G (!(serv1=1 & serv2=1) ) ]: true.
P>=1 [G (!(serv1=2 & serv2=2) ) ]: true.
P>=1 [G (!(serv1=3 & serv2=3) ) ]: true.
\end{verbatim}

All the properties hold for both versions of the model.

The relative responsiveness is modelled by using rewards.
Every time scheduling begins,
the rewards record how many active tasks are in the queue.
If the relative responsiveness is low, the tasks should spend relatively much time
in the queue, while if the relative responsiveness is high,
the tasks should spend relatively little time in the queue.
We find the reward by using cumulative rewards and queries.

The rewards code is as follows:

\begin{verbatim}
rewards "waiting_time"
  [schedule1] state1=1 & task1>0 : 1;
  [schedule1] state2=1 & task2>0 : 1;
  [schedule1] state3=1 & task3>0 : 1;
endrewards
\end{verbatim}

For probabilistic choice between servers:

\begin{verbatim}
R=? [C<=200] ~ 72.5112.
R=? [C<=1000] ~ 358.9560.
R=? [C<=10000] ~ 3581.4602
\end{verbatim}

For the first server only considering the tasks with the highest priority:

\begin{verbatim}
R=? [C<=200] ~ 72.4766.
R=? [C<=1000] ~ 358.8097.
R=? [C<=10000] ~ 3580.0578.
\end{verbatim}

The rewards are pretty similar. This fits with the intuition that since priorities
are independent from task length, making a server just serve jobs that have high priority
does not influence the responsiveness.

We can then consider what happens if we want the response time to be higher for
high-priority jobs. Will the model that have a server reserved for high-priority jobs
have a relatively lower response time than the model without such reservations for any servers?
We measure this by increasing the reward for high-priority jobs,
meaning that if the reward is relatively lower for one model than the other,
then the high-priority jobs spends relatively less time in the queue than in the queue
of the other model.

\begin{verbatim}
rewards "waiting_time"
  [schedule1] state1=1 & task1>0 : prio1;
  [schedule1] state2=1 & task2>0 : prio2;
  [schedule1] state3=1 & task3>0 : prio3;
endrewards
\end{verbatim}

For probabilistic choice between servers:

\begin{verbatim}
R=? [C<=200] ~ 106.9348.
R=? [C<=1000] ~ 528.8627.
R=? [C<=10000] ~ 5275.5521.
\end{verbatim}

For the first server only considering the tasks with the highest priority:

\begin{verbatim}
R=? [C<=200] ~ 105.3975.
R=? [C<=1000] ~ 520.8725.
R=? [C<=10000] ~ 5194.9664.
\end{verbatim}

The rewards are somewhat more different this time,
especially in the long run.
This fits with the intuition that if tasks with high priority
have a server reserved for them, they are going to spend relatively
less time in the queue than if they did not have a server reserved for them.

