
\section{A2}

\subsection{A2.1}

\subsubsection{a}

$
P = \begin{bmatrix}
0.0 & 0.4 & 0.0 & 0.6 \\
0.0 & 0.5 & 0.5 & 0.0 \\
0.4 & 0.0 & 0.2 & 0.4 \\
0.0 & 0.0 & 1.0 & 0.0 \\
\end{bmatrix}
$

$\iota_{init} = [1.0, 0.0, 0.0, 0.0]$

\subsubsection{b}

$\Theta_0 = \iota_{init} = [1.0, 0.0, 0.0, 0.0]$

$\Theta_1 = \iota_{init} \cdot P = [0.0, 0.4, 0.0, 0.6]$

$\Theta_2 = \iota_{init} \cdot P \cdot P = \Theta_1 \cdot P = [0.0, 0.2, 0.8, 0.0]$

$\Theta_3 = \iota_{init} \cdot P \cdot P \cdot P = \Theta_2 \cdot P = [0.32, 0.1, 0.26, 0.32]$

\subsubsection{c}

$x \cdot P = x$

$LS: sum(x) = 1$

$L0: x_0 = 0.4x_2$

$L1: x_1 = 0.4x_0 + 0.5x_1 \Leftrightarrow x1 = 0.8x_0$

$L2: x_2 = 0.5x_1 + 0.2x_2 + 1.0x_3$

$L3: x_3 = 0.6x_0 + 0.4x_2$

$ $

$x_1 = 0.4x_0 + 0.5x_1 \Leftrightarrow    (L1)$

$x_1 = 0.4\cdot0.4x_2 + 0.5x_1 \Leftrightarrow$

$x_1 = 0.32x_2$

$ $

$x_2 = 0.5x_1 + 0.2x_2 + 1.0x_3 \Leftrightarrow   (L2)$

$x_2 = 0.5x_1 + 0.2x_2 + 1.0(1 - x_0 - x_1 - x_2) \Leftrightarrow   (LS)$

$x_2 = 1 - x_0 - 0.5x_1 - 0.8x_2 \Leftrightarrow$

$x_2 = 1 - x_0 - 0.5\cdot0.8x_0 - 0.8x_2 \Leftrightarrow   (L1)$

$x_2 = 1 - 1.4x_0 - 0.8x_2 \Leftrightarrow$

$x_2 = 1 - 1.4\cdot0.4x_2 - 0.8x_2 \Leftrightarrow   (L0)$

$2.36x_2 = 1 \Leftrightarrow$

$x_2 = 1/2.36 \approx 0.423728814$

$x_0 = 0.4x_2 = 0.4\cdot0.423728814 \approx 0.169491525$

$x_1 = 0.8x_0 = 0.8\cdot0.169491525 \approx 0.13559322$

$x_2 \approx 0.423728814$

$x_3 = 0.6x0 + 0.4x_2 \approx 0.6\cdot0.169491525 + 0.4\cdot0.423728814 \approx 0.271186441$

$x \approx [0.169491525, 0.13559322, 0.423728814, 0.271186441]$

\subsection{A2.2}

The steady-state probabilities fits, except in regards to rounding
and limited precision.

\begin{verbatim}
dtmc

module somemodule
	// local state
	s : [0..3] init 0;
	
	[] s=0 -> 0.4 : (s'=1) + 0.6 : (s'=3);
	[] s=1 -> 0.5 : (s'=1) + 0.5 : (s'=2);
	[] s=2 -> 0.4 : (s'=0) + 0.2 : (s'=2) + 0.4 : (s'=3);
	[] s=3 -> 1.0 : (s'=2);
	
endmodule

system
  somemodule
endsystem
\end{verbatim}

\subsection{A2.3}

\subsubsection{a}

The probability of reaching a state where $a$ holds is
the probability of going directly to $s_1$,
plus the probability of staying one time step
in $s_0$ and then going to $s_1$.

$(1-p) + p(1-p)$

If $p$ is 0, the formula clearly holds.
The question is then what the upper bound for $p$ is.
To find that, we solve for the probability being 0.5:

$(1-p) + p(1-p) = 0.5 \Leftrightarrow$

$1 - p + p - p^2 = 0.5 \Leftrightarrow$

$0.5 = p^2 \Leftrightarrow$

$p = 0.5^{1 \over 2} \approx 0.707106781$

Thus, the values of $p$ that satisfy the formula is in the interval: $[0; 0.707106781]$.

\subsubsection{b}

The expected value is the probability for each outcome multiplied with the value of the outcome:

$\sum_{i=1}^\infty (1-p)p^{i-1}i )$

Since $(1-p)p^{i-1}$ is the geometric distribution,
the expected value can be found by:

$1 \over {1 - p}$

Note that we assume that p is less than 1, since else the expected value is undefined.

\subsubsection{c}

We modify the transition system thus:

$S = \{s_0, s_1, s_2\}$

$
P = \begin{bmatrix}
0.8 & 0.2 & 0.0 \\
0.0 & 0.9 & 0.1 \\
1.0 & 0.0 & 0.0 \\
\end{bmatrix}
$

$\iota_{init} = [1.0, 0.0, 0.0]$

$AP = \{\}$

$L = \{s_0 \to \{\}, s_1 \to \{\}, s_2 \to \{\}\}$

The reasoning is we can set up the transitions so that they fit with geometric distributions,
and thus in order to get an expected value of for instance 10,
we must have a probability of 0.1 to going on, and 0.9 to stay.

\subsection{A2.4}

\subsubsection{a}

$P_{\geq {17 \over 19}} [b U c]:$

$Sat(b) = \{s_1, s_4, s_5, s_6\}$

$Sat(c) = \{s_3, s_4, s_7\}$

$ $

$P_{=1} = \{s_3, s_4, s_7\}$

$P_{=0} = \{s_0, s_2\}$

$P_{=?} = \{s_1, s_5, s_6\}$

$
A = \begin{bmatrix}
0   & 0   & 1/4 \\
1/3 & 0   & 1/3 \\
0   & 1/2 & 0 \\
\end{bmatrix}
$

$b = [1/2, 1/3, 1/2]$

$x = A\cdot x + b$

$ $

$x_1 = 1/4x_6 + 1/2$

$x_5 = 1/3x_1 + 1/3x_6 + 1/3$

$x_6 = 1/2x_5 + 1/2$

Using equation solver:

$x_1 = 14/19$

$x_5 = 17/19$

$x_6 = 18/19$

The satisfying states:

$\{s_3, s_4, s_5, s_6, s_7\}$

\subsubsection{b}

$P_{>{1\over3}} [(b \vee c) U^{\leq2} (b \wedge c)]:$

$s_4$ is the only valid goal state.
Only starting states that can reach within 2 steps are $s_1$, $s_5$ and $s_6$.
$s_1$ clearly only has $1/4$ probability of reaching $s_4$ within 2 steps, so it is out.
$s_6$ clearly only has $1/6$ probability of reaching $s_4$ within 2 steps, so it is also out.
$s_5$ has probability $1/3 + 1/12$, which is higher than $1/3$, so it is in.
So, the states are $s_4$ and $s_5$.

$P_{\geq {1\over2}} [X \{s_4, s_5\}]: \{s_0, s_4, s_6\}$

The satisfying states are then $\{s_0, s_4, s_6\}$.

