
\section{A2}

TODO: Write.

\subsection{A2.1}

\subsubsection{a}

$S = \{s_0, s_1, s_2, s_3\}$

$\rightarrow = \{
  (s_0, s_1), (s_0, s_2),
  (s_1, s_1),
  (s_2, s_2), (s_2, s_3),
  (s_3, s_3)
\}$

$S_0 = \{s_0\}$

$AP = \{\Phi_1, \Phi_2\}$

$L = \{
  (s_0, \{\Phi_1\}),
  (s_1, \{\Phi_2\}),
  (s_2, \{\}),
  (s_3, \{\Phi_1, \Phi_2\})
\}$

\subsubsection{b}

i.: "For all paths, $\Phi_2$ eventually holds". False, since $\Phi_2$ never holds on the path $s_0{s_2}^w$.

ii.: "For all paths, $\Phi_2$ holds in the next state". False, since $\Phi_2$ does not hold
     in the second state of the initial path fragment $s_0s_2$.

iii.: "For at least one path, $\Phi_1$ eventually holds". True, since $\Phi_1$ holds in $s_0$.

iv.: "For all paths, $\Phi_1$ holds until $\Phi_2$". False, since $\Phi_2$ never holds on the path $s_0{s_2}^w$.

\subsection{A2.2}

\subsubsection{a}

Equivalent; reasoning by arguing for equivalence with initial path fragment.

The initial path fragment consists of an initial state $s_0$,
followed by $n$ states, $n \ge 1$, where the $n$'th state
is the first of the $n$ states to have $\Phi$ true.
Something like: $s_0 ( s_1 s_2 s_3 .. s_n ), s_n \models \Phi$.

Formula 1 is equivalent with the initial path fragment:
The formula says that there exists a path with a successor state,
and for the successor state there exists a path with $\Phi$
holding eventually.
Any witness for this formula would then consist of two path fragments;
one from the initial state to the next, and one from the next
to some state, with $\Phi$ holding in that state.
This fits exactly with the initial path fragment.

Formula 2 is equivalent with the initial path fragment:
The formula says that there exists a path where it eventually holds,
that there exists a path with a successor state with $\Phi$.
Any witness for this formula would then consist of two path fragments;
one from the initial state to some other state,
and one from the other state to the next, with $\Phi$ holding in that state.
This fits exactly with the initial path fragment.

By transitivity, formula 1 is equivalent with formula 2.

\subsubsection{b}

Equivalent; reasoning by arguing for equivalence with prefix computation tree, called $a$.

The computation tree is some tree, where for all leaves $\Phi$ holds,
and where all leaves are of depth at least 1.

Formula 1 and $a$:
Formula 1 says that for all paths, in the next state it must hold,
that for all paths $\Phi$ eventually holds.
Any witness is a prefix computation tree,
where for all the depth-1 nodes it holds,
that $\Phi$ holds for the leaves in their subtrees.
This fits exactly with $a$.

Formula 2 and $a$:
Formula 2 says that for all paths it must eventually hold,
that for all paths, in the next state $\Phi$ must hold.
Any witness is a prefix computation tree,
where for all leaves $\Phi$ holds,
and where all leaves are of depth at least 1.
This fits exactly with the prefix computation tree.

By transitivity, formula 1 is equivalent with formula 2.

\subsubsection{c}

Neither implies the other.

Transition system where formula 1 holds and formula 2 does not hold:

$S = \{s_0, s_1\}$

$\rightarrow = \{
  (s_0, s_1), (s_1, s_0)
\}$

$S_0 = \{s_0\}$

$AP = \{\Phi\}$

$L = \{
  (s_0, \{\Phi\})
\}$

Transition system where formula 2 holds and formula 1 does not hold:

$S = \{s_0, s_1, s_2\}$

$\rightarrow = \{
  (s_0, s_1), (s_0, s_2),
  (s_1, s_1),
  (s_2, s_2)
\}$

$S_0 = \{s_0\}$

$AP = \{\Phi\}$

$L = \{
  (s_1, \{\Phi\})
\}$

\subsubsection{d}

Equivalent; $\Phi_1$ and $\Phi_2$ must hold everywhere we can reach.
Another argument; distributive laws.

\subsubsection{e}

Formula 1 implies formula 2, but formula 2 does not imply formula 1.

Argument for formula 1 implies formula 2:

If formula 1 holds, then there exists at least one path,
where at some point at that path, both $\Phi_1$ and $\Phi_2$ holds.
Something like, $s_0s_1s_2..s_n$, and $\Phi_1$ and $\Phi_2$ holds in $s_n$.
But that also means formula 2 holds, since there exists a path
where at some point $\Phi_1$ holds, and likewise for $\Phi_2$.

Argument for formula 2 does not imply formula 1:

Transition system:

$S = \{s_0, s_1, s_2\}$

$\rightarrow = \{
  (s_0, s_1), (s_0, s_2),
  (s_1, s_1),
  (s_2, s_2)
\}$

$S_0 = \{s_0\}$

$AP = \{\Phi_1, \Phi_2\}$

$L = \{
  (s_1, \{\Phi_1\}),
  (s_2, \{\Phi_2\})
\}$

Formula 2 holds, but formula 1 does not.

\subsection{A2.3}

\subsubsection{a}

$EGF \Phi$

Not valid CTL+ formula (see page 426 of Principles of Model Checking),
but this CTL formula should be equivalent:

$\neg AF (AG \neg \Phi)$

The reasoning is as follows:
It is NOT true, that no matter which path is taken,
at some point, $\Phi$ will never again be true.
Basically, negating the exact opposite of "some path exists where
$\Phi$ holds infinitely often".


\subsubsection{b}

$A ( \Phi_1 U ( \Phi_2 \wedge O \Phi_3) )$

Maybe no.
Not valid CTL+ formula.

\subsubsection{c}

$E (F \Phi_1 \vee F \Phi_2)$

Valid CTL+ formula, so also possible to express in CTL:

$EF \Phi_1 \vee EF \Phi_2$

\subsubsection{d}

$A (G \Phi_1 \vee G \Phi_2)$

Valid CTL+ formula, so also possible to express in CTL:

$\neg EF ( \neg \Phi_1 \wedge EF \neg \Phi_2) \wedge
\neg EF ( \neg \Phi_2 \wedge EF \neg \Phi_1)$

Basically, counter-examples is that there exists a path
where first $\Phi_1$ fails at some point, and then $\Phi_2$ fails at some point,
or the other way round.
So, to write an equivalent CTL-formula,
require that the counter-examples may never be true.

\subsection{A2.4}

See appendix "A2.4".

The claims regarding the properties were verified to be correct.

