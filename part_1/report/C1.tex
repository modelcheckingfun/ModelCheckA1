\section{C}

\subsection{C1}

In this problem we need to modify the round-robin scheduler so it includes priorities. Furthermore
this scheduler should avoid starvation.
The Client module is unchanged. All the modifications are made in the scheduler.
First we include priorities and time into the scheduler to avoid starvation.
\begin{verbatim}
  priority1 : [1..2] init 1;
  priority2 : [1..2] init 1;

  time1 : [1..2] init 1;
  time2 : [1..2] init 1;
\end{verbatim}

When we create a job then the priority and time is set to either 1 or 2.
We have an index which points to, if either job1 or job2 should be executed.
If job1 is true and currentQueueIndex=1 and the time of the job is larger than 1 then we
subtract it with one time unit and execute the job.
If job1 is true and currentQueueIndex=1 and the time of the job is equal to 1, then we execute the job, 
assign currentQueueIndex to 2 and assign the time of the job to the priority.
We do this to ensure that job2 can be executed. We also do this vice versa with job2,
to ensure job1 can be executed.
\begin{verbatim}
  // Serve the current job and shift the index
  [serve1] job1 & currentQueueIndex=1 & time1>1 -> (time1'=time1-1);
  [serve1] job1 & currentQueueIndex=1 & time1=1 -> (currentQueueIndex'=2)
   & (time1'=priority1);
  [serve2] job2 & currentQueueIndex=2 & time2>1 -> (time2'=time2-1);
  [serve2] job2 & currentQueueIndex=2 & time2=1 -> (currentQueueIndex'=1)
   & (time2'=priority2);
\end{verbatim}
When we complete a job, then if job1 equals true and currentQueueIndex equals 1, then 
job1 is assigned false and currentQueueIndex is assigned to 2.
\begin{verbatim}
  // Complete any job that has finished
  [finish1] job1=true & currentQueueIndex=1 -> (job1'=false)
   & (currentQueueIndex'=2);
  [finish1] job1=true & currentQueueIndex!=1 -> (job1'=false);
  [finish2] job2=true & currentQueueIndex=2 -> (job2'=false)
   & (currentQueueIndex'=1);
  [finish2] job2=true & currentQueueIndex!=2 -> (job2'=false);
\end{verbatim}
The full modules can be seen in the Appendix.
The implications of avoiding starvation is that at some point a job will be executed.
We've specified some additional properties of the scheduler in CTL:

$AG(state1=1 \Rightarrow (AF(state1=0)))$ \newline
This formula means that for paths, it's always the case that whenever state1 equals 1,
then it implies that, for all paths eventually state1 equals 0.

$AG(state2=1 \Rightarrow (AF(state2=0)))$ \newline
This formula means that for paths, it's always the case that whenever state2 equals 1,
then it implies that, for all paths eventually state2 equals 0.

$AG(EF(state1=0))$ \newline
This formula means that for paths, it's always the case that, there exists eventually
a state where state1 equals 0.

$AG(EF(state2=0))$ \newline
This formula means that for paths, it's always the case that, there exists eventually
a state where state2 equals 0.

$AG(state1=1 \Rightarrow job1) \& (job1=>state1=1))$ \newline
This formula means that for paths, it's always the case that, whenever state1 equals 1,
then it implies job1 is true and whenever job1 is true, then it implies that
state1 equals 1.

$AG(state2=1 \Rightarrow job2) \& (job2=>state2=1))$ \newline
This formula means that for paths, it's always the case that, whenever state2 equals 1,
then it implies job2 is true and whenever job2 is true, then it implies that
state2 equals 1.