
\section{A1}

TODO: Write.

\subsection{A1.1}

\subsubsection{a}

For all paths, it should always hold on every step of the path,
if the client is active it should have its job somewhere in the queue.

\begin{verbatim}
P>=1 [ G ((state1=1) => (job1=1|job2=1)) ]
P>=1 [ G ((state2=1) => (job1=2|job2=2)) ]
\end{verbatim}

\subsubsection{b}

The properties hold according to PRISM.

\subsubsection{c}

The queue for the SRT scheduler is different.
"job1" being true means that a job for
client 1 is in the queue,
while "job2" being true means that a job for
client 2 is in the queue.

When client 1 is active,
it means that "state1" is true.
Since we require that when a client is active,
its job is in the scheduler's queue,
that means that "state1" implies "job1"
everywhere, or in other words, for all paths
always, "state1" implies "job1".
Repeat for client 2.

\begin{verbatim}
P>=1 [ G ((state1=1) => job1) ]
P>=1 [ G ((state2=1) => job2) ]
\end{verbatim}

\subsubsection{d}

The properties hold according to PRISM.

\subsection{A1.2}

\subsubsection{a}

The scheduler module should be modified
to cope with an extra clint, however the length of
the queue shouldn't be increased.

A new module for the new client must be created. This is
done by making a copy of the client1 module called client3.
Then the first and second job in the queue must be modified
to handle the new client. This is done by increasing the length
from 2 to 3. When placing a new job at the end of the queue, this
new job needs to be able to be placed by client3.

\begin{verbatim}
[create3] job2=0 -> (job2'=3);
\end{verbatim}

The job served at the head of the queue (job1) needs to be modified
so it corresponds to the job placed by client3 is executed.

\begin{verbatim}
[serve3] job1=3 -> true;
\end{verbatim}

The job at the head of the queue needs to finish if it's been placed
by client3

\begin{verbatim}
[finish3] job1=3 -> (job1'=0);
\end{verbatim}

Last but not least client3 is required to run in parallel as client2

\begin{verbatim}
system
  scheduler || client1 || client2 || client3
endsystem
\end{verbatim}

\subsubsection{b}

There's 214 reachable states in the model as we assume PRISM removes all non-reachable states

\subsubsection{c}

The scheduler cannot create a new job if the queue is full, because the scheduler
requires job2 to be 0 before it can place a new job at the end of the queue.

\subsubsection{d}

The properties hold according to PRISM.

\subsection{A1.4}

\subsubsection{a}

$\Phi$ holding for a transition system means that $\Phi$ holds
in all initial states.

$AG \Phi$ holding for a transition system means that $\Phi$ holds
for all states reachable from the initial states.

\subsubsection{b}

In PRISM 4.0.*, a property is verified for all the initial states.
meaning that the meanings of $\Phi$ and $AG \Phi$ are different

\subsubsection{c}

Give $s$ an atomic proposition with a unique name,
maybe $initialstate$, and then define $\Phi '$ thus:

$initialstate \implies \Phi$

This will ensure that $\Phi$ is only checked for $s$,
despite checking $\Phi '$ for all states.

\subsubsection{d}

Define $\Phi '$ thus:

$AG \Phi$

