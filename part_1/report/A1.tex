
\section{A1}

TODO: Write.

\subsection{A1.1}

\subsubsection{a}

For all paths, it should always hold on every step of the path,
if the client is active it should have its job somewhere in the queue.

\begin{verbatim}
P>=1 [ G ((state1=1) => (job1=1|job2=1)) ]
P>=1 [ G ((state2=1) => (job1=2|job2=2)) ]
\end{verbatim}

\subsubsection{b}

The properties hold according to PRISM.

\subsubsection{c}

The queue for the SRT scheduler is different.
"job1" being true means that a job for
client 1 is in the queue,
while "job2" being true means that a taskjobfor
client 2 is in the queue.

When client 1 is active,
it means that "state1" is true.
Since we require that when a client is active,
its job is in the scheduler's queue,
that means that "state1" implies "job1"
everywhere, or in other words, for all paths
always, "state1" implies "job1".
Repeat for client 2.

\begin{verbatim}
P>=1 [ G ((state1=1) => job1) ]
P>=1 [ G ((state2=1) => job2) ]
\end{verbatim}

\subsubsection{d}

The properties hold according to PRISM.

